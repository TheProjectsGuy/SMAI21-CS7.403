% !TeX root = main.tex

\section[Q3: Bayes Theorem]{Bayes Theorem}

% Answer A
\subsection{A: Prior and Posterior probabilities}

The \textbf{prior} probability is the probability of the event \textit{before} considering the evidence and the \textbf{posterior} probability is the probability of the event \textit{after} considering the evidence (likelihood and evidence).

The Bayes Theorem states how a hypothesis should be updated, given some new evidence

\begin{equation}
    \mathbb{P} \left ( H \mid E \right ) = \frac{\mathbb{P} \left ( E \mid H \right ) \, \mathbb{P} (H)}{\mathbb{P} (E)}
    \label{eq:bayes-theorem}
\end{equation}

Where $\mathbb{P} \left ( H \mid E \right )$ is the \textit{Posterior} probability, $\mathbb{P} \left ( E \mid H \right )$ is the likelihood, $\mathbb{P} (E)$ is the evidence, and $\mathbb{P} (H)$ is the \textit{Prior} probability.

% Answer B
\subsection{B: Bayes Rule example}

Let $F$ denote the event of having the flu and $S$ denote the event of having the symptoms currently being experienced (headache and a soar throat).

Given that $P(S | F) = 0.90$, that is, the probability of getting the symptoms $S$ given that one has the flu is $90\%$.

Given that $P(F) = 0.05$, that is, the probability of getting the flu is $5\%$.

Given that $P(S) = 0.20$, that is, the chances of getting the symptoms in the population.

We need to estimate the probability of having flu given the symptoms, that is, $P(F|S)$.

\begin{align}
    P(F|S) &= \frac{P(S|F) \times P(F)}{P(S)}
    \nonumber \\
    &= \frac{0.90 \times 0.05}{0.20} = \frac{9}{40} = 0.225 = 22.5 \%
\end{align}

Therefore, there is a $22.5 \%$ chance of having the flu, given the symptoms being experienced.

However, this result may be flawed, given that there is knowledge of a friend being sick with flu. Since data about transmissibility and interactions is not available, no assumptions are included in the evidence (only symptoms used here).
